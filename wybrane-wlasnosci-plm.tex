\section{Wybrane własności przestrzeni liniowo-metrycznych}

\begin{df}
  Niech $(Y, d)$ będzie przestrzenią liniowo-metryczną. Metrykę $d$ nazwiemy niezmienniczą, gdy zachodzi warunek:
  \[d(x+a, y+a) = d(x,y)\ \forall x, y, a \in Y\] 
\end{df}


\begin{fact}
  Na dowolnej przestrzeni liniowo-metrycznej $Y$ z metryką $d$ można wprowadzić metrykę niezmienniczą.
  
  \begin{proof}(Szkic)
  
  Przestrzeń liniowo-metryczna z operacją $+$ stanowi grupę topologiczną z topologią zadaną przez metrykę, a więc spełniającą pierwszy aksjomat przeliczalności oraz własność $T_2$. Z twierdzenia Birkhoffa-Kakutaniego istnieje więc na niej metryką lewostronnie niezmiennicza.
  \end{proof}
\end{fact}

\begin{note}
  Niech $(Y, d)$ przestrzeń liniowo-metryczna z metryką niezmienniczą. Wówczas metryka $\rho := \min(d, 1)$ jest ograniczoną, niezmienniczą metryką równoważną.
  \begin{proof}
    Równoważność wynika z równości małych kul, tzn. $\forall n \in \mathbb{N}_1: B_\rho(0, \frac{1}{n}) = B_d(0, \frac{1}{n})$. Metryka $\rho$ jest oczywiście ograniczona przez 1. Niezmienniczość:
    
    \[\rho(x+a, y+a) = \min(d(x+a, y+a), 1) = \min(d(x,y), 1) = \rho(x,y)\]
  \end{proof}
\end{note}


Z uwagi na powyższy fakt na przestrzeniach liniowo-metrycznych będziemy rozważali tylko ograniczone przez $1$ metryki niezmiennicze.

\begin{df}
  Niech $(Y, d)$ będzie przestrzenią liniowo-metryczną. Metrykę $d$ nazwiemy rosnącą po promieniach, gdy:
  \[d(t_1 x, 0) < d(t_2 x, 0)\ \forall 0 \leq t_1 < t_2\ \forall x \neq 0\]
  
  Gdy powyższa nierówność jest prawdziwa w wersji słabej ($\leq$) metrykę taką nazwiemy słabo rosnącą po promieniach.
\end{df}

\begin{lem}
  Aby metryka $d$ była rosnąca po promieniach wystarczy aby warunek $d(t_1 x, 0) < d(t_2 x, 0)$ zachodził dla liczb wymiernych $0 \leq t_1 < t_2$ i $x \neq 0$.
  
  \begin{proof}
    Niech $0 \leq a < b$ będą liczbami rzeczywistymi.
    Wówczas istnieją silnie monotoniczne ciągi $(a_n)_{n=1}^\infty$ oraz $(b_n)_{n=1}^\infty$ liczb wymiernych takie, że $a_n \downarrow a$ i $b_n \uparrow b$ oraz $a_1 < b_1$. Zatem $\forall n > 1$:
    \[d(a_n x, 0) < d(a_1 x, 0) < d(b_1 x, 0) < d(b_n x, 0)\]
    
    Co po przejściu granicznym daje:
    \[d(a x, 0) \leq d(a_1 x, 0) < d(b_1 x, 0) \leq d(b x, 0)\]
    
    Zatem $d(ax, 0) < d(bx, 0)$. 
  \end{proof}
\end{lem}


\begin{thm}[Eidelheit-Mazur] \label{thm:eidelheit-mazur}
  Niech $(Y, d)$ przestrzeń liniowo-metryczna. Wówczas istnieje ograniczona przez $1$, niezmiennicza metryka $\rho$ równoważna $d$ rosnąca po promieniach.
  \begin{proof}
    Określamy dla $w > 0$ metrykę $\rho_w(x, y) := \sup_{0 \leq t \leq w} d(tx, ty)$. $\rho_w$ jest metryką, jako skończone supremum metryk ograniczonych przez $1$. Z niezmienniczości $d$ wynika niezmieniczość $\rho_w$. Co więcej, jest metryką równoważną $d$.
    
    Istotnie:
    \begin{align*}
      \rho_w(x_n, x) \to 0 &\implies 0 \leq d(w x_n, wx) \leq \rho_w(x_n,x)  \to 0 \\
      &\implies d(wx_n, wx) \to 0 \\
      &\implies d(x_n, x) \to 0
    \end{align*}
    gdzie ostatnia implikacja wynika z ciągłości mnożenia (mnożymy przez $w^{-1}$).
    
    Z drugiej strony, niech $d(x_n, x) \to 0$ oraz $\varepsilon > 0$. Zdefiniujmy odwzorowanie:
    \[m: [0,w] \times Y \ni (t,x) \to d(tx, 0) \in [0,\infty)\]
    
    Z ciągłości wynika $m$ wynika, że dla każdego $t$ istnieje $V_t$ otoczenie $t$ oraz $U_t$ otoczenie $0$ w $Y$ takie, że $m(s, y) < \varepsilon$ o ile tylko $s \in V_t$ i $y \in U_t$.
    
    Ze zwartości odcinka wybieramy z pokrycia $\{V_t\}_{t \in [0,w]}$ skończone podpokrycie $(V_{t_i})_{i=1}^{k}$
    
    Biorąc $U := \bigcap_{i=1}^k U_{t_i}$ otrzymujemy otwarte otoczenie $0$ o własności:
    \[t \in [0,w], x \in U,\mbox{ to } m(t,x) = d(tx, 0) < \varepsilon\]
    
    Ustalmy $n_0$ tak, aby dla każdego $n \geq n_0$ było $x_n - x \in U$. Ale wtedy:
    \[\rho_w(x_n, x) = \rho_w(x_n - x, 0) = \sup_{0 \leq t \leq w} d(t(x_n - x), 0) < \varepsilon\]
    
    A więc $\rho_w(x_n, x) \to 0$, co dowodzi równoważności $\rho_w$ i $d$.
    
    Zauważmy jeszcze, że $\rho_w$ jest metryką słabo rosnącą po promieniach. Rzeczywiście, niech $0 \leq t_1 < t_2$, $x \in Y$. Mamy:
    \[\rho_w(t_1 x, 0) = \sup_{0 \leq t \leq w} d(t \cdot t_1 x, 0) \leq \sup_{0 \leq t \leq w} d(t \cdot t_2 x, 0) = \rho_w(t_2 x, 0)\]
    
    Ułóżmy wszystkie liczby wymierne dodatnie w ciąg $(w_i)_{i=1}^\infty$. Definiujemy wynikową metrykę jako:
    \[\rho(x, y) := \sum_{i=1}^\infty \frac{1}{2^i} \rho_{w_i}(x, y)\]
    
    Zbieżność powyższego szeregu wynika z ograniczoności metryk $\rho_w$ przez $1$, co więcej $\rho(x,y) \leq 1$. Z niezmienniczości metryk $\rho_w$ wynika niezmienniczość $\rho$. Ponadto metryka $\rho$ jest równoważna metryce $d$.
    
    Istotnie, metryka $\rho$ jest większa niż $\frac{1}{2} \rho_{w_1}$, która jest równoważna metryce $d$, a więc $\rho(x_n, x) \to 0$ pociąga $d(x_n, x) \to 0$.
    
    Z drugiej strony, niech $d(x_n, x) \to 0$ i $\varepsilon > 0$. Z równoważności $d$ i $\rho_w$ mamy $\rho_w(x_n, x) \to 0$.
    
    Ustalmy $k \in \mathbb{N}_1$ takie, że $\sum_{i=k+1}^\infty \frac{1}{2^i} < \frac{\varepsilon}{2}$.
    Ustalmy teraz $n_0 \in \mathbb{N}_1$ takie, że:
    \[\forall i \in [1,k]\ \forall n > n_0: \frac{1}{2^i} \rho_{w_i}(x_n, x) \leq \frac{\varepsilon}{2k}\]
    
    Wówczas $\forall n > n_0$:
    \[\rho(x_n, x) \leq \sum_{i=i}^k \frac{1}{2^i} \rho_{w_i}(x_n, x) + \sum_{i=k+1}^\infty \frac{1}{2^i} \leq k \cdot \frac{\varepsilon}{2k} + \frac{\varepsilon}{2} = \varepsilon\]
    a więc $\rho(x_n, x) \to 0$
    
    Pokażemy teraz, że metryka $\rho$ jest rosnąca po promieniach. Ustalmy $0 \leq t_1 < t_2$ oraz $x \in Y$. Z poprzedniego lematu wiemy, że wytarczy rozważyć jedynie $t_1, t_2 \in \mathbb{Q}$. Ze słabego wzrostu po promieniach metryk $\rho_{w_i}$ wynika słaby wzrost po promieniach metryki $\rho$. Wystarczy więc wykazać, że dla pewnego $i$ zachodzi $\rho_{w_i}(t_1 x, 0) < \rho_{w_i}(t_2 x, 0)$. Przypuścmy dla dowodu niewprost, że $\forall i: \rho_{w_i}(t_1 x, 0) = \rho_{w_i}(t_2 x, 0)$. Weźmy ciąg malejący do zera ciąg liczb wymiernych: $(t_1^i t_2^{-i})_{i=0}^\infty$ i przyjrzyjmy się metrykom $\rho_w$ związanym z kolejnymi wyrazami tego ciągu. Najpierw zanotujmy:
    \begin{align*}
      \rho_{t_1^i t_2^{-i}}(t_1 x, 0) = \sup_{0 \leq t \leq t_1^i t_2^{-i}} d(t \cdot t_1 x, 0) =
      \sup_{0 \leq t \leq t_1^{i+1} t_2^{-(i+1)}} d(t \cdot t_2 x, 0) = \rho_{t_1^{i+1} t_2^{-(i+1)}}(t_2 x, 0)
    \end{align*}
    Skąd otrzymujemy ciąg równości:
    \begin{align*}
      \rho_1(t_2 x, 0) &= \rho_1(t_1 x, 0) = \\
      \rho_{t_1 t_2^{-1}}(t_2 x, 0) &= \rho_{t_1 t_2^{-1}}(t_1 x, 0) = \\
      \rho_{t_1^2 t_2^{-2}}(t_2 x, 0) &= \rho_{t_1^2 t_2^{-2}}(t_1 x, 0) = \\
      \cdots
    \end{align*}
    A więc
    \begin{align*}
      d(t_2 x, 0) \leq \sup_{0 \leq t \leq 1} d(t \cdot t_2 x, 0) &= \rho_1(t_2 x, 0) \\
      &= \rho_{t_1^i t_2^{-i}}(t_1 x, 0) = \sup_{0 \leq t \leq t_1^i t_2^{-i}} d(t \cdot t_1 x, 0)
    \end{align*}
    Z równoważności metryki $d$ i $\rho_{t_1}$ wynika, że skoro $t_1^i t_2^{-i} \to 0$ -- a więc i $d(t_1^i t_2^{-i} x, 0) \to 0$ -- to:
    \[
      \sup_{0 \leq t \leq t_1^i t_2^{-i}} d(t \cdot t_1 x, 0)
      = \sup_{0 \leq t \leq t_1} d(t \cdot t_1^i t_2^{-i} x, 0)
      = \rho_{t_1}(t_1^i t_2^{-i} x, 0)
    \]
    również dąży do zera. Zatem $d(t_2 x, 0) = 0$, co w konsekwencji daje $x = 0$. Zatem jedynym przypadkiem, gdy nie zachodzi wzrost po promieniach jest $x = 0$, zatem $\rho$ jest metryką rosnącą po promieniach.
  \end{proof}
\end{thm}

\begin{lem} \label{lem:ball-homeomorphism}
  Niech $Y$ będzie przestrzenią liniowo-metryczną. Wówczas przestrzeń $Y$ z metryką rosnącą po promieniach $d$ spełnia warunek: dla każdego $\varepsilon > 0$ kula $U_\varepsilon := B_d(0, \varepsilon)$ jest homeomorficzna z $Y$.
  
  \begin{proof}
    Weźmy metrykę $d$ na $Y$ rosnącą po promieniach z twierdzenia Eidelheita-Mazura. Niech $0 < t_1 < t_2$. Pokażemy, że:
    \[
      \cl{t_1 U_\varepsilon} \subset t_2 U_\varepsilon
    \]
    
    Istotnie, niech $y \in \cl{t_1 U_\varepsilon}$. Wówczas istnieje ciąg $(y_n)_{n=1}^\infty \in U_\varepsilon^{\mathbb{N}_1}$ taki, że $t_1 y_n \to y$. Wynika stąd, że $d(y_n, 0) < \varepsilon$ a po przejściu granicznym $d(\frac{y}{t_1}, 0) \leq \varepsilon$. Ale z monotoniczności metryki $d$ mamy: $d(\frac{y}{t_2}, 0) < d(\frac{y}{t_1}, 0) \leq \varepsilon$ a więc $y \in t_2 U_\varepsilon$.
    
    Definiujemy odwzorowanie:
    \[
      \lambda: Y \ni y \to \inf \{t \geq 0\ |\ y \in t U_\varepsilon\} \in [0, \infty)
    \]
    
    Podobnie jak w przypadku funkcjonału Minkowskiego pokażemy, że $\lambda$ spełnia warunek:
    
    \[
      y \in t_0 U_\varepsilon \iff \lambda(y) < t_0
    \]
    
    Istotnie, niech $y \in t_0 U_\varepsilon$
    Wówczas z otwartości zbioru $U_\varepsilon$ i ciągłości mnożenia istnieje $t_1 < t_0: y \in t_1 U_\varepsilon$.
    A więc $\lambda(y) \leq t_1 < t_0$.
    Z drugiej strony, jeśli $\lambda(y) < t_0$, to z definicji infimum istnieje $t_1 < t_0: y \in t_1 U_\varepsilon$, ale $\cl{t_1 U_\varepsilon} \subset t_0 U_\varepsilon$, więc $y \in t_0 U_\varepsilon$.

    Wykażemy ciągłość odwzorowania $\lambda$.
    Ustalmy $y_0 \in Y$ oraz $\delta > 0$.
    Niech $U := (\lambda(y_0) + \delta) U_\varepsilon \setminus \cl{(\lambda(y_0)-\delta) U_\varepsilon}$.
    
    Po pierwsze zauważmy, że $y_0 \in U$.
    Istotnie, skoro $\lambda(y_0) < \lambda(y_0) + \delta$, to $y_0 \in (\lambda(y_0) + \delta) U_\varepsilon$.
    Z drugiej strony, gdyby $y_0 \in \cl{(\lambda(y_0) - \delta) U_\varepsilon} \subset (\lambda(y_0) - \frac{\delta}{2}) U_\varepsilon$. Ale wtedy $\lambda(y_0) < \lambda(y_0) - \frac{\delta}{2}$.
    Zatem $U$ jest otwartym otoczeniem $y_0$.
    
    Następnie zauważmy, że $y \in (\lambda(y_0) + \delta) U_\varepsilon$, to $\lambda(y) < \lambda(y_0) + \delta$ oraz $y \not\in \cl{(\lambda(y_0) - \delta) U_\varepsilon}$, to $\lambda(y) \geq \lambda(y_0) - \delta$, gdyż w przeciwnym razie $\lambda(y) < \lambda(y_0) - \delta$ a więc $y \in (\lambda(y_0) - \delta) U_\varepsilon$.
    
    Podsumowując: $y \in U$, to $\lambda(y_0) - \delta < \lambda(y) < \lambda(y_0) + \delta$, a więc z dowolności $\delta$ wykazaliśmy ciągłość $\lambda$.
    
    Warto jeszcze zaznaczyć, że $\lambda(sy) = \inf \{t \geq 0\ |\ sy \in t U_\varepsilon\} = s\inf \{t \geq 0\ |\ y \in t U_\varepsilon\} = s \lambda(y)\ \forall s \geq 0, y \in Y$.
    
    Definiujemy oczekiwany homeomorfizm jako:
    \[
      h: Y \ni y \to \frac{y}{\max\left(1, \lambda(y) + \frac{1}{2}\right)} \in U_\varepsilon
    \]
    Bezpośrednio ze wzoru wynika, że $h$ jest funkcją ciągłą. Aby sprawdzić, że jest poprawnie określona należy zbadać, czy $h(y) \in U_\varepsilon$. Sprawdzimy więc:
    
    \begin{align*}
      \lambda\left(\frac{y}{\max\left(1, \lambda(y) + \frac{1}{2}\right)}\right) =
      \frac{\lambda(y)}{\max\left(1, \lambda(y) + \frac{1}{2}\right)} \leq
      \frac{\lambda(y)}{\lambda(y) + \frac{1}{2}} < 1
    \end{align*}
    A więc $h(y) \in 1 U_\varepsilon = U_\varepsilon$.
    
    Aby pokazać, że $h$ jest homeomorfizmem wyliczymy funkcję odwrotną:
    \[
      \frac{y}{\max\left(1, \lambda(y) + \frac{1}{2}\right)} = z
    \]
    Dla $\lambda(y) \geq \frac{1}{2}$ mamy $y = z$, dla $\lambda(y) \leq \frac{1}{2}$ natomiast 
    \[
      \frac{y}{\lambda(y) + \frac{1}{2}} = z
    \]
    Skąd:
    \begin{equation}
      \label{a1}
      \lambda(z) = \frac{\lambda(y)}{\lambda(y) + \frac{1}{2}}
    \end{equation}
    Po prostych przekształceniach dostajemy:
    \[
      \lambda(y) = \frac{\frac{1}{2} \lambda(z)}{1 - \lambda(z)}
    \]
    Zatem
    \[
      y = z\left(\frac{\frac{1}{2} \lambda(z)}{1 - \lambda(z)} + \frac{1}{2}\right) = \frac{z}{2(1-\lambda(z))}
    \]
    Zauważmy jeszcze, że $\lambda(y) \leq \frac{1}{2} \iff \lambda(z) \leq \frac{1}{2}$. Istotnie, jeśli $\lambda(y) \leq \frac{1}{2}$, to ze wzoru \eqref{a1} natychmiast wynika, że $\lambda(z) < 1$. A dla $\lambda(z) < 1$ mamy:
    \begin{align*}
      \lambda(y) &\leq \frac{1}{2} \\
      1 &\geq \lambda(y) + \frac{1}{2} = \frac{1}{2(1-\lambda(z))} \\
      2(1-\lambda(z)) &\geq 1 \\
      \frac{1}{2} &\geq \lambda(z)
    \end{align*}
    Mamy więc wzór na funkcję odwrotną do $h$:
    \[
      h^{-1}(z) = 
      \begin{cases}
        \frac{z}{2(1-\lambda(z))},&\lambda(z) \leq \frac{1}{2} \\
        z,&\lambda(z) \geq \frac{1}{2}
      \end{cases}
    \]
    Ale wzór ten możemy również zapisać jako:
    \[
      z \to \frac{z}{2(1-\min(\lambda(z), \frac{1}{2}))}
    \]
    Zatem $h^{-1}$ jest funkcją ciągłą.
  \end{proof}
\end{lem}

W dalszym ciągu będziemy rozważali przestrzenie liniowo-metryczne z metryką niezmienniczą, ograniczoną przez $1$ i rosnącą po promieniach.

\begin{lem}
  Niech $Y$ będzie przestrzenią liniowo-metryczną, $M$ przestrzenią parazwartą. Wówczas następujące wartunki są równoważne:
  \begin{enumerate}
   \item[(i)] każdy punkt $x \in M$ ma otoczenie homeomorficzne ze zbiorem otwartym w $Y$
   \item[(ii)] każdy punkt $x \in M$ ma otoczenie homeomorficzne z przestrzenią $Y$
   \item[(iii)] każdy punkt $x \in M$ ma bazę otoczeń homeomorficznych z $Y$
  \end{enumerate}

  \begin{proof}
    Implikacje $(iii) \implies (ii) \implies (i)$ są jasne.
    
    Pokażemy, że $(i) \implies (iii)$. Niech $x \in M$ oraz niech $f: M \supset U \to V \subset Y$ będzie mapą w otoczeniu $x$. Niech $\varepsilon_0 > 0$ będzie taki, że $B(f(x), \varepsilon_0) \subset V$. Niech $\varepsilon < \varepsilon_0$. Z lematu \ref{lem:ball-homeomorphism} istnieje homeomorfizm $g: B(f(x), \varepsilon) \to Y$, bo kula $B(0, \varepsilon)$ w przestrzeni liniowo-topologicznej jest homeomorficzna z kulą $B(f(x), \varepsilon)$. Niech $f_0$ będzie homeomorfizmem powstałym z zawężenia $f$ w obrazie do $B(f(x), \varepsilon)$ i dziedzinie do $G_\varepsilon := f^{-1}(B(f(x), \varepsilon))$. Wówczas $f_0 \circ g$ jest homeomorfizmem między $G_\varepsilon$ a $Y$.
    
    Skoro $\{B(f(x), \varepsilon)\}_{0 < \varepsilon < \varepsilon_0}$ jest bazą w punkcie $f(x)$, to $\{G_\varepsilon\}_{0 < \varepsilon < \varepsilon_0}$ jest bazą w punktcie $x$.
  \end{proof}
\end{lem}

\begin{lem}
  Niech $(Y, d)$ będzie przestrzenią liniowo-metryczną. Wówczas kula $B_r$ o promieniu $r$ jest drogowo spójna.
  \begin{proof}
    Niech $x \in B_r$. Określamy:
    \[
      \gamma: [0,1] \ni t \to tx \in B_r
    \]
    Należy zwrócić uwagę, że odwzorowanie to jest poprawnie określone, bo metryka $d$ rośnie po promieniach, zatem: $d(0, tx) \leq d(0, x) < 1$ dla $t \in [0,1]$, a więc $tx \in B_r$.
    
    Z ciągłości działań w przestrzeni $Y$ wynika ciągłość $\gamma$. Używając odwzorowania $\gamma$ możemy połączyć drogą każdy punkt kuli z $0$, zatem sklejając dwie takie drogi możemy połączyć każde dwa punkty kuli.
  \end{proof}
\end{lem}

\begin{lem} \label{lem:connectedness}
  Niech $(Y, d)$ będzie przestrzenią liniowo-metryczną. Wówczas $Y$ jest drogowo spójna.
  \begin{proof}
    Z poprzedniego lematu i niezmienniczości metryki wnioskujemy, że $Y$ jest lokalnie drogowo spójna. Z lematu \ref{lem:local-connectedness} wynika, że $Y$ jest drogowo spójna.
  \end{proof}
\end{lem}

\begin{lem}
  Niech $Y$ będzie nietrywialną przestrzenią liniowo-metryczną, a $M$ spójną rozmaitością topologiczną modelowaną na $Y$. Wówczas:
  \[
    \wght Y = \wght M
  \]
  \begin{proof}
    Niech $\mathcal U_0$ będzie pokryciem $M$ złożonym ze zbiorów homeomorficznych ze zbiorami otwartymi w $Y$. Z parazwartości wpisujemy w $\mathcal U_0$ pokrycie otwarte, lokalnie skończone $\mathcal U$. Dla każdego punktu $x \in M$ weźmy bazę otoczeń $\{U_n^x\}_{n \in \mathbb{N}_1}$ punktu $x$ o tej własności, że dla każdego $n \in \mathbb{N}_1$ mamy $\card\{U \in \mathcal U\ |\ U \cap U_n^x \neq \emptyset\} < \aleph_0$ oraz, że każdy ze zbiorów $U_n^x$ jest homeomorficzny z kulą w $Y$. Każdy punkt ma otoczenie przecinające tylko skończenie wiele zbiorów $\mathcal U$ z lokalnej skończoności, aby uzyskać $U_n^x$ wystarczy więc otoczenie to zmniejszyć aby obraz był kulą o promieniu mniejszym niż $\frac{1}{n}$. Biorąc wówczas $\mathcal B := \{U_n^x\ |\ n \in \mathbb{N}_1, x \in M\}$ dostajemy bazę $M$.
    
    Ustalmy $U \in \mathcal U$. Wtedy $\mathcal B_U := \{U \cap B\ |\ B \in \mathcal B\}$ jest bazą $U$. Z lematu \ref{lem:base-small} wybieramy z $\mathcal B_U$ podzbiór $\mathcal D_U$, który jest bazą przestrzeni $U$ o własności:
    \[
      \card \mathcal D_U \leq \wght U
    \]
    Ale $U$ jest homeomorficzne ze zbiorem otwartym w $Y$, zatem $\wght U = \wght Y$. Nierówność $\wght U \leq \wght Y$ jest jasna, druga z nich wynika z tego, że obraz $U$ przez homeomorfizm jest zbiorem otwartym, więc zawiera w sobie jakąś kulę. A kula w $Y$ jest homeomorficzna z $Y$. Zatem $\wght Y = \wght U$.
    
    Dalej weźmy $\mathcal D := \bigcup_{U \in \mathcal U} \mathcal D_U$.
    Niech $D \in \mathcal D$. Wówczas istnieje jedynie skończenie wiele $U \in \mathcal U$ przecinających $D$. Ale w każdym z $U$ jest co najwyżej $\wght Y$ zbiorów z $\mathcal D$, mianowice $\mathcal D_U$. Zatem $D$ przecina co najwyżej $\aleph_0 \cdot \wght Y = \wght Y$ innych zbiorów z $\mathcal D$. Nazwijmy to własnością $(\star)$.
    
    Ustalamy $x \in M$, oraz $D \in \mathcal D$. Niech $y \in D$. Łączymy $x$ z $y$ drogą $\gamma$. Zwarty obraz $\gamma$ pokrywamy skończoną liczbą zbiorów z $\mathcal D$ przecinających się z $\im\gamma$ oraz zbiorem $D$. Otrzymujemy w ten sposób skończoną liczbę zbiorów spójnych (jako przeciwobrazy kul, które są spójne) pokrywających $\im\gamma$, których suma jest również zbiorem spójnym. Zauważmy, że takich struktur jest tylko $\wght Y$. 
    
    Istotnie, prześledźmy jak powstaje dowolny zbiór tej postaci. Weźmy zbiór przykrywający $x$, następnie wszystkie zbiory, które go przecinają. Ze względu na własność $(\star)$ istnieje tylko $\wght Y$ kandydatów na te zbiory odległe o $1$. Postępując idukcyjnie, tzn. mając zbiory odległe o $n$ - kandydatów na zbiory odległe o $n+1$, czyli przecinające się ze zbiorami odległymi o $n$ jest również, ze względu na $(\star)$ tylko $\wght Y$. Zatem istnieje tylko $\wght Y$ takich struktur. Nie zwiększymi również liczby tych struktur jeśli jeden ze zbiorów będzie oznaczony.
    
    Biorąc rzutowanie z rodziny wszystkich struktur na oznaczony zbiór otrzymujemy suriekcję ze zbioru mocy $\wght Y$ na $\mathcal D$, zatem $\mathcal D$ jest bazą $M$ mocy $\wght Y$.
  \end{proof}
\end{lem}

\begin{df}
  Niech $r > 0$. Podzbiór $A$ przestrzeni metrycznej $M$ nazwiemy $r$-rozstrzelonym, jeśli:
  \[
    \forall x, y \in A: d(x,y) \geq r
  \]
  Podzbiór $A$ przestrzeni metrycznej $M$ nazwiemy $r$-siecią, jeśli:
  \[
    \forall x \in M\ \exists a \in A: d(x,a) < r
  \]
\end{df}

\begin{lem} \label{lem:anti-net}
  Niech $M$ będzie przestrzenią metryczną, $r > 0$. Wówczas w $M$ istnieje maksymalny zbiór $r$-rozstrzelony. Ponadto, maksymalny zbiór $r$-rozstrzelony jest $r$-siecią.
  \begin{proof}
    Zbiór wszystkich zbiorów $r$-rozstrzelonych jest niepusty, ponieważ $\emptyset$ jest zbiorem $r$-rozstrzelonym i częściowo uporządkowany przez inkluzję. Weźmy łańcuch zbiorów $r$-rozstrzelonych. Wówczas suma wszystkich elementów wszystkich zbiorów łańcucha jest również zbiorem $r$-rozstrzelonym. Istotnie, gdyby tak nie było, to istniałyby dwa punkty oddalone od siebie o mniej niż $r$. Ale wówczas istniałby element łańcucha zawierający oba z nich. Sprzeczność. Lemat Kuratowskiego-Zorna kończy dowód dając element maksymalny $A$.
    
    Gdyby zbiór ten nie był $r$-siecią istniałby punkt $x$ oddalony od dowolnego punktu zbioru $A$ o co najmniej $r$. Zatem zbiór $A \cup \{x\}$ byłby istotnie większym zbiorem $r$-rozstrzelonym.
  \end{proof}
\end{lem}

\begin{lem}
  Niech $Y$ będzie nietrywialną przestrzenią liniowo-metryczną. Wówczas istnieje $A \subset Y$ o mocy $\card A \geq \wght Y$ oraz dyskretna rodzina kul $\{\cl B(y, r_y)\}_{y \in A}$.
  
  \begin{proof}
    Najpierw znajdziemy w $Y$ promień $r > 0$ oraz przeliczalną dyskretną rodzinę kul. Niech $y_0 \in Y\setminus\{0\}$. Kładziemy $r := \frac{1}{4} d(y_0,0)$. Wówczas rodzina kul: $\{B(ny_0, r)\}_{n \in \mathbb{N}_1}$ jest przeliczalna i dyskretna. Istotnie, dobrym otoczeniem punktu $y \in Y$ jest kula $B(y,r)$, ponieważ odległość dwu różnych środków $n_1 y$ i $n_2 y$, dla $n_1 < n_2$ wynosi:
    \[
      d(n_2 y_0, n_1 y_0) = d((n_2 - n_1)y_0, 0) \geq d(y_0, 0) = 4r
    \]
    Oznaczmy przez $B := B(0,r)$. Weźmy dla $n > 0$ z lematu \ref{lem:anti-net} maksymalny zbiór $\frac{r}{n}$ rozstrzelony w $B$ i oznaczmy go przez $A_n$. Niech $A := \bigcup_{n=1}^\infty A_n$. Zauważmy, że zbiór $\{B(y, \frac{r}{n})\ |\ y \in A_n, n \in \mathbb{N}_1\}$ jest bazą $B$. Istotnie, jeśli $y \in B$ i $\varepsilon > 0$ takie, że $B(y,\varepsilon) \subset B$. Wówczas biorąc $n$ takie, że $2 \cdot \frac{r}{n} < \varepsilon$ znajdujemy z lematu \ref{lem:anti-net} punkt $a$ zbioru $A_n$ oddalony od $y$ o mniej niż $\frac{r}{n}$. Wówczas kula $B(a, \frac{r}{n})$ zawiera $y$ i zawiera się w $B(y,\varepsilon)$.
    
    Wynika stąd, że $\card A \geq \wght B = \wght Y$.
    
    Wystarczy teraz każdą rodzinę kul związaną ze swoim zbiorem $\frac{r}{n}$-rozstrzelonym w osobnej kuli $B(ny_0, r)$, tzn. szukana rodzina kul wygląda następująco: $\{\cl B(ny_0 + y, \frac{r}{2n})\ |\ n \in \mathbb{N}_0, y \in A_n\}$.
  \end{proof}
\end{lem}

