\section{Funkcje sterujące}

\begin{df}
  Niech $X$ będzie przestrzenią metryczną. Niech: 
  \[
    d_n(x, y) := \sum_{i=1}^n \frac{1}{2^i} d(x_i, y_i)
  \]
  Na przestrzeni $X^n := \prod_{i=1}^n X_i$, gdzie $X_i := X$, będziemy zawsze rozważać metrykę $d_n$.
  
  Przestrzeń $Y := \prod_{i=1}^\infty X$ będziemy zawsze rozważali z topologią Tichonowa i zgodną z nią metryką $d_\infty$.
  
  Ustalamy projekcję na pierwsze $n$ współrzędnych:
  \[
    s_n: Y \ni y \rightarrow (y_1, \ldots, y_n) \in X^n
  \]
  Ustalamy $x_0 \in X$. Niech $\star_n := (x_0)_{i=n+1}^\infty \in X^n$.
\end{df}

\begin{df}
Niech $X$ będzie dowolnym zbiorem. Zbiorem $n$-bazowym nazwiemy zbiór $A$ zawarty w $Y := \prod_{i=1}^\infty X$, który da się zapisać jako $A = B \times \prod_{i=n+1}^\infty X$ dla pewnego $n$ i pewnego $B \subset \prod_{i=1}^n X$. Zbiór $B$ będziemy nazywać ($n$-)podstawą zbioru ($n$-)bazowego.
\end{df}

\begin{prop}[Własności zbiorów bazowych] \mbox{} % break line here
\begin{enumerate}
  \item Zbiór $n$-bazowy jest jednocześnie $m$-bazowy dla $n \leq m$
  \item Dla zbioru bazowego $A$ o podstawie $B$ zachodzi: $\bd A = \bd B \times Y_n$, gdzie $Y_n := \prod_{i=n+1}^\infty X$.
  \begin{proof}
    Policzmy:
    \begin{align*} 
      \bd A = \bd(B \times Y_n) = \cl{B \times Y_n} \cap \cl{Y \setminus (B \times Y_n)} &= \\
      \cl{B \times Y_n} \cap \cl{(X^n \setminus B) \times Y_n} &= \cl{B} \times Y_n \cap \cl{X^n \setminus B} \times Y_n = \bd{B} \times Y_n
    \end{align*}
    Dla uproszczenia zapisu będziemy ze wzlędu na naturalne utożsamienie $Y = X^n \times Y_n = X^n \times Y$ pisać: $\bd A = \bd B \times Y$.
  \end{proof}
  \item Domknięcie zbioru $n$-bazowego jest zbiorem $n$-bazowym
  \begin{proof}
    Niech $A$ będzie zbiorem bazowym o podstawie $B$. Wówczas:
    \[
      \cl A = \cl{B \times Y} = \cl B \times Y
    \]
    A więc $\cl A$ jest zbiorem bazowym o podstawie $\cl B$.
  \end{proof}
\end{enumerate}
\end{prop}

\begin{df}
Niech $X$ będzie przestrzenią topologiczną, niech $Y := \prod_{n=1}^\infty X$. Funkcją sterującą nazwiemy odwzorowanie ciągłe $\lambda: Y \rightarrow [1,\infty]$ takie, że:
\[\lambda(x) < n+1 \mbox{ i } s_n(x) = s_n(y) \Rightarrow \lambda(x) = \lambda(y)\]
\end{df}

\begin{note} \label{note:steering-example}
Przykłady funkcji sterujących:
  \begin{enumerate}
    \item \label{steering-example-const} funkcja stała, $\lambda(x) := \mbox{const}$
    \item \label{steering-example-max} jeśli $\lambda_1$ i $\lambda_2$ są funkcjami sterującymi, to $\lambda(y) := \max(\lambda_1(y), \lambda_2(y))$ jest też funkcją sterującą
    \item \label{steering-example-sum} jeśli $\lambda_1$ i $\lambda_2$ są funkcjami sterującymi, to $\lambda(y) := \lambda_1(y) + \lambda_2(y)$ też jest funkcją sterującą
  \begin{proof}
     Pokażemy \ref{steering-example-max}. Niech $\lambda(y) < n+1$ i $s_n(x) = s_n(y)$. Wówczas $\lambda_1(y) < n+1$ oraz $\lambda_2(y) < n+1$, więc $\lambda(x) = \max(\lambda_1(x), \lambda_2(x)) = \max(\lambda_1(y), \lambda_2(y)) = \lambda(y)$.
     
     Aby wykazać \ref{steering-example-sum} niech znowu $\lambda(y) < n+1$ i $s_n(x) = s_n(y)$. Wówczas ze względu na nieujemność funkcji sterujących $\lambda_1(y) < n+1$ i $\lambda_2(y) < n+1$. A więc $\lambda(x) = \lambda_1(x) + \lambda_2(x) = \lambda_1(y) + \lambda_2(y) = \lambda(y)$.
  \end{proof}
  \end{enumerate}
\end{note}


\begin{thm}[O poprawianiu funkcji sterujących]
\label{thm:steering-function}
Niech $X$ będzie przestrzenią metryczną. Niech $Y := \prod_{n=1}^\infty X$. Niech $A$ będzie domkniętym podzbiorem $Y$ a $\lambda: Y \rightarrow [1,\infty]$ funkcją sterującą. Wówczas funkcję sterującą $\lambda$ można poprawić do funkcji sterującej $\rho$ tak, aby:

\begin{gather*}
 \rho|A = \lambda|A \\
 \rho(y) < \infty\ \forall y \in Y \setminus A
\end{gather*}

\begin{proof}
  Na początku zdefiniujmy następujące zbiory:
  \[
    U_n := \left\{y \in Y\ \middle|\ d_n(s_n y, s_n A) > \frac{1}{n}\right\} \cup \{y \in Y\ |\ \lambda(y) < n+1\}
  \]
  Zauważmy, że $U_n$ są zbiorami $n$-bazowymi. Istotnie, niech $s_n(y) = s_n(x)$:
  \begin{itemize}
    \item jeśli $d_n(s_n y, s_n A) > \frac{1}{n}$, to $d_n(s_n x, s_n A) > \frac{1}{n}$
    \item jeśli $\lambda(y) < n+1$, to z własności sterowania mamy również $\lambda(x) < n+1$
  \end{itemize}
  Zatem jeśli tylko $y \in U_n$ i $s_n(y) = s_n(x)$ to również $x \in U_n$, co oznacza $n$-bazowość zbioru $U_n$. Zapiszmy więc $U_n$ w postaci: $U_n := V_n \times Y$, gdzie $V_n$ jest $n$-podstawą.
  
  Określamy również zbiory:
  \[
    A_n := \cl{s_n({\{y \in A\ |\ \lambda(y) < n+1\}})}
  \]
  Pokażemy dla nich trzy istotnie własności, mianowicie:
  \begin{enumerate}[(1)]
    \item \label{A_n-1} jeśli $x \in A_n$, to $d_n(x, s_n A) = 0$
    \item \label{A_n-2} jeśli $s_n y \in A_n$, to $\lambda y = \lambda(s_n y,\star_{n+1})$
    \item \label{A_n-3} jeśli $x \in A_n$, to $\lambda(x,\star_{n+1}) \leq n+1$
  \end{enumerate}
  Wykażemy \ref{A_n-1}. Niech $x \in A_n$. Wówczas:
  \[
    \exists y_k \in A, \lambda(y_k) < n+1: s_n y_k \to x
  \]
  Skąd:
  \[
    d_n(x, s_n y_k) \to 0
  \]
  A więc:
  \[
    d_n(x, s_n A) = 0
  \]
  Wykażemy \ref{A_n-2}. Niech $s_n y \in A_n$. Podobnie:
  \[
    \exists z_k \in A, \lambda(z_k) < n+1: s_n z_k \to s_n y
  \]
  Wtedy z warunku sterowania:
  \[
    \lambda(s_n z_k, y_{n+1}, y_{n+2}, \ldots) = \lambda(s_n z_k,\star_{n+1})
  \]
  Przechodząc do granicy otrzymujemy:
  \[
    \lambda(y) = \lambda(s_n y,\star_{n+1})
  \]
  Wykażemy \ref{A_n-3}. Niech $x \in A_n$. Znowu:
  \[
    \exists y_k \in A, \lambda(y_k) < n+1: s_n y_k \to x
  \]
  Ze sterowania $\lambda$:
  \[
    n+1 > \lambda(y_k) = \lambda(s_n y_k, \star_{n+1})
  \]
  Przechodząc do granicy:
  \[
    n+1 \geq \lambda(x, \star_{n+1})
  \]
  
  Skonstruujemy teraz indukcyjnie rodzinę funkcji:
  \[\rho_n: \cl{U_n} \rightarrow [1, n+1]\]
  spełniających następujące warunki:
  \begin{enumerate}[(a)]
    \item \label{induction-1} $\rho_n(s_n y, z) = \lambda(y)$ o ile $s_n(y) \in A_n$, $z \in Y$
    \item \label{induction-2} $\rho_n \supset \rho_{n-1}, \qquad \rho_0 := \emptyset$
    \item \label{induction-3} $\rho_n y \geq n$ dla $y \in \cl{U_n \setminus U_{n-1}},\qquad U_0 := \emptyset$
    \item \label{induction-4} $\rho_n y = n+1$ dla $y \in \bd U_n$
  \end{enumerate}
  Skorzystamy także z pomocniczych funkcji
  \[
    \sigma_n: \cl{V_n} \rightarrow [1,n+1]
  \]
  Określając później $\rho_n$ jako przedłużenie $\sigma_n$ w sposób produktowy:
  \[
    \rho_n(y) := \sigma_n(s_n(y))
  \]
  %
  Konstrukcja:
  \[
    \sigma_1: \cl{V_1} \rightarrow [1,2]
  \]
  Określamy $\sigma_1$ następującymi wzorami:
  \[
  \sigma_1(x) := 
    \begin{cases}
      2,&\mbox{dla }x \in \bd V_1 \\
      \lambda(x, \star_2),& \mbox{dla }x \in A_1 
    \end{cases}
  \]
  Zauważmy, że obie te definicje są ze sobą zgodne. Istotnie, weźmy $x \in \bd V_1 \cap A_1$. Wówczas $(x, \star_2) \in \bd V_1 \times Y = \bd U_1$ i zachodzi jeden z warunków:
  \begin{itemize}
    \item $d_1(s_1 (x, \star_2), s_1 A) = 1$
    \item $\lambda(x, \star_2) = 2$
  \end{itemize}
  Warunek pierwszy z pewnością nie zachodzi, gdyż $x \in A_1$ pociąga $d_1(x, s_1 A) = 0$, co wyklucza równość $d_1(x, s_1 A) = 1$.
  Warunek drugi oznacza natomiast zgodność obu wzorów.
  
  Funkcję $\sigma_1$ określoną na zbiorze domkniętym $\bd V_1 \cup A_1$ przedłużamy z twierdzenia Tietzego na zbiór $\cl{V_1}$ z zachowaniem ciągłości.
  
  Zgodnie z zapowiedzią definiujemy więc:
  \[
    \rho_1(y) := \sigma_1(s_1(y))\ \forall y \in \cl{U_1}
  \]
  Sprawdźmy, że funkcja $\rho_1$ spełnia warunki \ref{induction-1}--\ref{induction-4}:
  \begin{enumerate}
   \item[\ref{induction-1}] Niech $s_1 y \in A_1$. Wówczas z warunku \ref{A_n-2} mamy $\lambda y = \lambda(s_1 y,\star_{2})$. A więc $\rho_1(s_1 y,z) = \sigma_1 s_1 y = \lambda(s_1 y,\star_2) = \lambda(y)$
   \item[\ref{induction-2}] Warunek jest pustospełniony
   \item[\ref{induction-3}] $\rho_1 \geq 1$
   \item[\ref{induction-4}] Jeśli $y\in\bd U_1$, to $s_1 y\in\bd V_1$, $\rho_1(y) = \sigma_1(s_1 y) = 2$
  \end{enumerate}
  %
  Załóżmy teraz, indukcyjnie, że mamy już określoną funkcję $\rho_{n-1}$. Zdefiniujemy:  
  \[
    \sigma_n': \cl{V_n \setminus (V_{n-1} \times X)} \rightarrow [n,n+1]
  \]
  Określając:
  \[
  \sigma_n'(x) := 
    \begin{cases}
      n,&\mbox{dla }x \in \bd V_{n-1} \times X \\
      n+1,&\mbox{dla }x \in \bd V_n \\
      \lambda(x, \star_{n+1}),&\mbox{dla }x \in A_n
    \end{cases}
  \]
  %
  Sprawdźmy, że funkcja ta jest poprawnie określona, tj. na ewentualnych przecięciach wypisanych zbiorów domkniętych zdefiniowana jest tymi samymi wzorami oraz $\sigma_n' \geq n$.
  
  Zauważmy, że:
  \[
    \bd V_n\ \cap\ (\bd V_{n-1} \times X) = \emptyset
  \]
  ponieważ:
  \[
    (\bd V_n \times Y) \cap (\bd V_{n-1} \times Y) = \bd U_n \cap \bd U_{n-1} = \emptyset
  \]
  bo $\cl{U_{n-1}} \subset U_n$.
  
  Zachodzi również zgodność pozostałych wzorów. Niech $(x, z) \in (\bd V_{n-1} \times X) \cap A_n$. Wówczas $(x, \star_n) \in \bd U_{n-1}$ a więc, jak poprzednio, $\lambda(x, z, \star_{n+1}) = \lambda(x, \star_n) = n$ - co daje zgodność wzorów lub $d_{n-1}(x, s_{n-1} A) = \frac{1}{n-1}$ - co z uwagi na \ref{A_n-1} wyklucza należenie do $A_n$, ponieważ $(x, z) \in A_n$ wymusza $d_{n}((x, z), s_{n} A) = 0$, a więc i $d_{n-1}(x, s_{n-1} A) = 0$.
  
  Podobnie pokazujemy zgodność $\bd V_n$ z $A_n$. Niech $x \in \bd V_n$, skąd $(x,\star_{n+1})\in\bd U_n$ a więc $d_n(s_n(x,\star_{n+1}), s_n A) = d_n(x,s_n A) = \frac{1}{n}$ lub $\lambda(x,\star_{n+1}) = n+1$. W pierwszym przypadku jednak dostajemy sprzeczność z \ref{A_n-1}. W drugim natomiast wzory są zgodne.
  
  Należy jeszcze sprawdzić, że zgadza się przeciwdziedzina. Niech więc $x \in A_n \cap \cl{V_n\setminus(V_{n-1}\times X)}$. Z własności \ref{A_n-3} mamy $\lambda(x,\star_{n+1}) \leq n+1$. Gdyby jednak $\lambda(x,\star_{n+1}) < n$, to $(x,\star_{n+1})\in U_{n-1}$, więc $s_{n-1} x\in V_{n-1}$, czyli $x\in V_{n-1}\times X$, sprzeczność.
  
  Tak określoną funkcję $\sigma_n'$ przedłużamy z twierdzenia Tietzego na całą postulowaną dziedzinę z zachowaniem przeciwdziedziny $[n,n+1]$.
  
  Ostatecznie, aby otrzymać $\sigma_n$ doklejamy do $\sigma_n'$ funkcję następującą:
  \[
    \sigma_{n-1}'': \cl{V_{n-1}\times X}\ni x \to \sigma_{n-1}s_{n-1}x \in [1,n],
  \]
  która na przecięciu dziedzin $\dom \sigma_n' \cap \dom \sigma_{n-1}'' = \cl{V_{n-1}\times X} \cap \cl{V_n \setminus (V_{n-1}\times X)} =\bd({V_{n-1}\times X}) = \bd{V_{n-1}}\times X$, osiąga $n$ (bo z poprzedniego kroku indukcji $\sigma_{n-1}$ na $\bd{V_{n-1}}$ ma dokładnie $n$) podobnie jak $\sigma_n$.
  
  Funkcję $\rho_n$ definiujemy jak poprzednio:
  \[
    \rho_n y := \sigma_n s_n y
  \]
  
  Sprawdzamy warunki \ref{induction-1}--\ref{induction-4}:
  \begin{enumerate}
    \item[\ref{induction-1}] Niech $s_n y \in A_n$. Jeśli $s_n y\in V_{n-1}\times X$, to $y\in U_{n-1}$. Ze względu na \ref{A_n-1} mamy więc $\lambda(y) < n$, a zatem $s_{n-1}y \in A_{n-1}$. Wówczas:
    \[
      \rho_n(y) = \sigma_n(s_n y) = \sigma_{n-1}''(s_n y) = \sigma_{n-1}(s_{n-1}y) = \rho_{n-1}(y) = \lambda(y)
    \]
    Jeśli $s_n y\not\in V_{n-1}\times X$, to zachodzi drugi wzór:
    \[
      \rho_n(y) = \sigma_n(s_n y) = \sigma_n'(s_n y) = \lambda(s_n y, \star_{n+1}) = \lambda(y),
    \]
    gdzie ostatnia równość wynika z \ref{A_n-2}.
   \item[\ref{induction-2}] Niech $y\in\cl{U_{n-1}} = \cl{V_{n-1}}\times Y$. Mamy:
   \[
     \rho_n(y) = \sigma_n s_n y = \sigma_{n-1}'' s_n y = \sigma_{n-1} s_{n-1} y = \rho_{n-1} y
   \]
   \item[\ref{induction-3}] Niech $y \in \cl{U_n \setminus U_{n-1}} = \cl{V_n \setminus (V_{n-1}\times X)}\times Y$, zatem:
   \[
     \rho_n(y) = \sigma_n s_n y = \sigma_n' s_n y \geq n,
   \]
   ponieważ $\im \sigma_n' \subset [n,n+1]$.
   \item[\ref{induction-4}] Jeśli $y\in\bd U_n = \bd V_n \times Y$, to:
   \[
     \rho_n y = \sigma_n s_n y = \sigma_n' s_n y = n+1
   \]
  \end{enumerate}
  To kończy nasze postępowanie indukcyjne.

  Spójrzmy teraz na jakim zbiorze pozostaje nam określić funkcję: $Y \setminus \bigcup_{i=1}^\infty \cl{U_i} = Y \setminus \bigcup_{i=1}^\infty U_i$. Oczywiście $\bigcup_{i=1}^\infty \{y\ |\ \lambda(y) < n+1\} = \{y\ |\ \lambda(y) < \infty\}$. Przyjrzyjmy się uważniej drugiej części, tj.: $\bigcup_{i=1}^\infty \{y\ |\ d_n(s_n y, s_n A) > \frac{1}{n}\}$. Twierdzimy, że jest ona równa $Y \setminus A$.
  
  Istotnie. Niech $d_n(s_n y, s_n A) \leq \frac{1}{n}\ \forall n$. Istnieje wówczas ciąg $(a_n)_{n=0}^\infty \in A^\mathbb{N}$, taki że:
  \[
    d_n(s_n y, s_n a_n) < \frac{2}{n}\ \forall n \in \mathbb{N}_1
  \]
  Niech $\varepsilon > 0$, niech $n_0$ będzie tak duże, by $\sum_{i=n_0}^\infty \frac{1}{2^i}< \frac{\varepsilon}{3}$ oraz $\frac{2}{n_0} < \frac{\varepsilon}{3}$. Wówczas dla dowolnego $n \geq n_0$:
  \begin{align*}
    d(y, a_n) \leq
    d(y, (s_n y, \star_{n+1})) &+ d((s_n y, \star_{n+1}), (s_n a_n, \star_{n+1})) \\
    &+ d((s_n a_n, \star_{n+1}), a_n) \leq \frac{\varepsilon}{3} + d_n(s_n y, s_n a_n) + \frac{\varepsilon}{3} \leq \varepsilon
  \end{align*}
  Co oznacza, że $a_n \rightarrow y$ i tym samym $y \in \cl{A} = A$. Zatem jedynym miejscem poza $\bigcup_{i=1}^\infty U_i$ jest $\{y \in A\ |\ \lambda(y) = \infty\}$.
  
  Określamy więc pożądaną funkcję sterującą $\rho$ jako:
  \[\rho(x) := 
    \begin{cases}
      \rho_n(x),&\mbox{dla } x \in U_n \\
      \infty,&\mbox{ w przeciwnym razie}
    \end{cases}
  \]
  %
  Wykażemy teraz, że funkcja $\rho$ ma postulowane własności:
  \begin{enumerate}
   \item funkcja $\rho$ jest poprawnie określona, co wynika bezpośrednio z \ref{induction-2}
   \item ciągłość $\rho$ na $U_i$ wynika z ciągłości $\rho_i$. Jeśli natomiast $x\in Y\setminus\bigcup_{i=1}^\infty U_i$, to na $Y\setminus\cl{U_{n-1}}$ otoczeniu $x$ mamy: $\exists k\geq n: y\in U_{k} \setminus \cl{U_{k-1}}$. Zatem z \ref{induction-3}: $\rho(y) = \rho_k(y) \geq k \geq n$.
   \item $x\not\in A$, to $\exists n\in\Ni: x\in U_n$, więc $\rho(x) = \rho_n(x) < \infty$
   \item zgodność z $\lambda$ na $A$. Niech $x\in A$:
   \begin{enumerate}[1$^\circ$]
    \item $\lambda(x) = \infty$, to $x\not\in U_n$, więc $\rho(x) = \infty$
    \item $\lambda(x) < \infty$, to $\exists n\in\Ni: \lambda(x) < n+1$, zatem $x \in U_n$ oraz $s_n x\in A_n$, co daje z warunku~\ref{induction-1}: $\rho(x) = \rho_n(x) = \lambda(x)$
   \end{enumerate}
   \item $\rho$ jest funkcją sterującą. Istotnie, niech $\rho(y) < n+1$ i $s_n(y) = s_n(x)$. Wówczas z warunku~\ref{induction-3}: $y \in U_n$ i tym samym $x \in U_n$, bo $U_n$ jest $n$-bazowy. A wtedy $\rho(y) = \rho_n(y) = \sigma_n(s_n y) = \sigma_n(s_n x) = \rho_n(x) = \rho(x)$
  \end{enumerate}
  
  Ostatecznie skonstruowaliśmy więc funkcję sterującą $\rho$ o żądanych własnościach.
\end{proof}
\end{thm}


\begin{cor} \label{cor:steering-finite}
  Niech $A$ będzie domkniętym podzbiorem $X$. Wówczas istnieje funkcja sterująca $\lambda$ taka, że $\lambda(y) = \infty$ wtedy i tylko wtedy gdy $y \in A$.
  
  \begin{proof}
    Niech $\mu: Y \ni y \rightarrow \infty \in [1, \infty]$ będzie funkcją sterującą. Z poprzedniego twierdzenia poprawiamy funkcję $\mu$ na $A$, otrzymując $\lambda$ o żądanych własnościach: $\lambda|A = \mu|A = \infty$ oraz $\lambda(y) < \infty$ dla $y \in Y \setminus A$.
  \end{proof}
\end{cor}

\begin{lem} \label{lem:elementary-inequality}
  Dla każdej liczby naturalnej większej lub równej $1$ zachodzi elementarna nierówność:
  \[4(n+1) \leq 2^{8n}\]
  
  \begin{proof}
    Dla $n = 1$ nierówność przyjmuje postać: $4(1+1) = 8 \leq 256 = 2^8$.
    Załóżmy, że nierówność jest prawdziwa dla $n \geq 1$. Wówczas:
    
    \[4(n+2) \leq 4(n+1) + 4 \leq 2^{8n} + 4 \leq 2^{8n} + 2^{8n} = 2^{8n + 1} \leq 2^{8(n+1)}\]
    
    Co kończy dowód indukcyjny.
  \end{proof}
\end{lem}

\begin{cor} \label{cor:steering-dist}
  Niech $A$ będzie domkniętym podzbiorem $X$. Wówczas istnieje funkcja sterująca $\rho$ taka że $\rho(y) = \infty$ wtedy i tylko wtedy gdy $y \in A$ oraz:
  
  \[2^{-\rho(y)} \leq \frac{1}{4}d(y,A)\ \forall y \in Y \setminus A\]
  
  \begin{proof}
    Przyjmijmy oznaczenia dowodu wniosku \ref{cor:steering-finite}, w szczególności niech $\lambda$ będzie skonstruowaną tam funkcją sterującą. Określamy: $\rho := 8 \cdot \lambda$. Z własności funkcji sterujących widzimy, że $\rho$ jest funkcją sterującą jako suma funkcji sterujących.
    Powtarzając dowód twierdzenia \ref{thm:steering-function} z uwagi na $\mu = \infty$, mamy w tym przypadku $U_n = \{y \in Y\ |\ d_n(s_n y, s_n A) > \frac{1}{n}\}$.
    
    Niech $y \in Y \setminus A$, istnieje więc $n$ takie, że $y \in U_{n+1}$. Mamy wtedy:
    
    \[\frac{1}{n+1} < d_{n+1}(s_{n+1}(y), s_{n+1} A)  \leq d(y, A)\]
    
    Z drugiej strony z warunku \ref{induction-3} otrzymujemy $\lambda(y) \geq n$, czyli $\rho(y) \geq 8n$.
    
    Zatem lemat \ref{lem:elementary-inequality} daje:
    \[
      2^{-\rho(n)} \leq 2^{-8n} \leq \frac{1}{4}\frac{1}{n+1} \leq \frac{1}{4}d(y,A),
    \]
    co należało pokazać.
  \end{proof}
\end{cor}
