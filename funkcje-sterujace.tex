\section{Funkcje sterujące}


\begin{df}
Zbiorem $n$-bazowym nazwiemy zbiór $A$ zawarty w $Y$, który da się zapisać jako $A = B \times \prod_{i=n+1}^\infty X_i$ dla pewnego $n$ i pewnego $B \subset \prod_{i=1}^n X_i$. Zbiór $B$ będziemy nazywać ($n$-)podstawą zbioru ($n$-)bazowego.
\end{df}


\begin{note}[Własności zbiorów bazowych] \mbox{} % break line here
\begin{itemize}
  \item Zbiór $n$-bazowy jest jednocześnie $m$-bazowy dla $n \leq m$
  \item Dla zbioru bazowego $A$ o podstawie $B$ zachodzi: $\partial A = \partial B \times Y$.
  \begin{proof}
    \[\partial A = \partial(B \times Y) = \cl{B \times Y} \cap \cl{Y \setminus (B \times Y)} = \cl{B \times Y} \cap \cl{(Y \setminus B) \times Y} = \cl{B} \times Y \cap \cl{Y \setminus B} \times Y = \partial{B} \times Y\]
  \end{proof}
  \item Domknięcie zbioru $n$-bazowego jest zbiorem $n$-bazowym
\end{itemize}
\end{note}


\begin{df}
Funkcją sterującą nazwiemy odwzorowanie ciągłe $\lambda: Y \rightarrow [1,\infty]$ takie, że:
\[\lambda(x) < n+1 \mbox{ i } s_n(x) = s_n(y) \Rightarrow \lambda(x) = \lambda(y)\]
\end{df}

\begin{note}
Przykłady funkcji sterujących:
  \begin{itemize}
    \item dowolna funkcja stała, $\lambda(x) := \mbox{const}$
    \item jeśli $\lambda_1$ i $\lambda_2$ są funkcjami sterującymi, to $\lambda(y) := \max(\lambda_1(y), \lambda_2(y))$ jest też funkcją sterującą
  \begin{proof}
    Niech $\lambda(y) < n+1$ i $s_n(x) = s_n(y)$. Wówczas $\lambda_1(y) < n+1$ oraz $\lambda_2(y) < n+1$, więc $\lambda(x) = \max(\lambda_1(x), \lambda_2(x)) = \max(\lambda_1(y), \lambda_2(y)) = \lambda(y)$.
  \end{proof}
  \end{itemize}
\end{note}


\begin{thm}[O poprawianiu funkcji sterujących]
\label{thm:steering-function}
Niech $A$ będzie domkniętym podzbiorem $Y$ a $\lambda: Y \rightarrow [1,\infty]$ funkcją sterującą. Wówczas funkcję sterującą $\lambda$ można poprawić do funkcji sterującej $\rho$ tak, aby:

\begin{gather}
 \rho|A = \lambda|A \\
 \rho(y) < \infty\ \forall y \in Y \setminus A
\end{gather}

\begin{proof}
  
  Na początku zdefiniujmy następujące zbiory:
  
  \[U_n := \{y \in Y\ |\ d_n(s_n y, s_n A) > \frac{1}{n}\} \cup \{y \in Y\ |\ \lambda(y) < n+1\}\]
  
  Zauważmy, że $U_n$ są zbiorami $n$-bazowymi. Istotnie, niech $s_n(y) = s_n(x)$:
  \begin{itemize}
    \item jeśli $d_n(s_n y, s_n A) > \frac{1}{n}$, to $d_n(s_n x, s_n A) > \frac{1}{n}$
    \item jeśli $\lambda(y) < n+1$, to z własności sterowania mamy również $\lambda(x) < n+1$
  \end{itemize}
  Zatem jeśli tylko $y \in U_n$ i $s_n(y) = s_n(x)$ to również $x \in U_n$, co oznacza $n$-bazowość zbioru $U_n$.
  
  Zapiszmy więc $U_n$ w postaci: $U_n := V_n \times Y$, gdzie $V_n$ jest $n$-podstawą.
  
  Określamy również zbiory:
  \[A_n := \cl{s_n({\{y \in A\ |\ \lambda(y) < n+1\}})}\]
  
  Zauważmy, że jeśli $x \in A_n$, to $d_n(x, s_n A) = 0$. Istotnie. Niech $x \in A_n$. Wówczas:
  \[\exists y_k \in A, \lambda(y_k) < n+1:\ s_n y_k \rightarrow x\]
  Skąd:
  \[d_n(x, s_n y_k) \rightarrow 0\]
  A więc:
  \[d_n(x, s_n A) = 0\]
  
  Skonstruujemy indukcyjnie rodzinę funkcji:
  \[\rho_n: \cl{U_n} \rightarrow [1, n+1]\]
  spełniających następujące warunki:
  \begin{enumerate}[(a)]
    \item \label{induction-1} $\rho_n(s_n y, z) = \lambda(y)$ o ile $s_n(y) \in A_n$, $z \in Y$
    \item \label{induction-2} $\rho_n \supset \rho_{n-1}$
    \item \label{induction-3} $\rho_n y \geq n$ dla $y \in \cl{U_n \setminus U_{n-1}}$
    \item \label{induction-4} $\rho_n y = n+1$ dla $y \in \partial U_n$
  \end{enumerate}
  
  Skorzystamy także z pomocniczych funkcji
  \[\sigma_n: \cl{V_n} \rightarrow [1,n+1]\]
  
  Określając później $\rho_n$ jako przedłużenie $\sigma_n$ w sposób produktowy:
  \[\rho_n(y) := \sigma_n(s_n(y))\]

  Konstrukcja:
  \[\sigma_1: \cl{V_1} \rightarrow [1,2]\]
  
  Określamy $\sigma_1$ następującymi wzorami:
  
  \[
  \sigma_1(x) := 
    \begin{cases}
      2,&\mbox{dla }x \in \partial V_1 \\
      \lambda(x, \star_2),& \mbox{dla }x \in A_1 
    \end{cases}
  \]
  
  Zauważmy, że obie te definicje są ze sobą zgodne. Istotnie, weźmy $x \in \partial V_1 \cap A_1$. Wówczas $(x, \star_2) \in \partial V_1 \times Y = \partial U_1$. Ale wówczas zachodzi jeden z warunków:
  
  \begin{itemize}
    \item $d_1(s_1 (x, \star_2), s_1 A) = 1$
    \item $\lambda(x, \star_2) = 2$
  \end{itemize}

  Warunek pierwszy z pewnością nie zachodzi, gdyż $x \in A_1$ pociąga $d_1(x, s_1 A) = 0$, co wyklucza równość $d_1(x, s_1 A) = 1$.
  Warunek drugi oznacza natomiast zgodność obu wzorów.
  
  Funkcję $\sigma_1$ określoną na zbiorze domkniętym $\partial V_1 \cup A_1$ przedłużamy z twierdzenia Tietzego na zbiór $\cl{V_1}$ z zachowaniem ciągłości.
  
  Zgodnie z zapowiedzią definiujemy więc:
  \[\rho_1(y) := \sigma_1(s_1(y))\ \forall y \in \cl{U_1}\]
  
  Funkcja ta w sposób jasny spełnia wszystkie stawiane warunki.
  
  Załóżmy teraz, indukcyjnie, że mamy już określoną funkcję $\rho_{n-1}$. Zdefiniujemy:
  
  \[\sigma_n: \cl{V_n \setminus (V_{n-1} \times X)} \rightarrow [1,n+1]\]
  
  Określając:
  \[
  \sigma_n(x) := 
    \begin{cases}
      n,&\mbox{dla }x \in \partial V_{n-1} \times X \\
      n+1,&\mbox{dla }x \in \partial V_n \\
      \lambda(x, \star_{n+1}),&\mbox{dla }x \in A_n
    \end{cases}
  \]

  Sprawdźmy, że funkcja ta jest poprawnie określona, tj. na ewentualnych przecięciach wypisanych zbiorów domkniętych zdefiniowana jest tymi samymi wzorami.
  
  Zauważmy, że:
  \[\partial V_n\ \cap\ (\partial V_{n-1} \times X) = \emptyset\]
  ponieważ:
  \[(\partial V_n \times Y) \cap (\partial V_{n-1} \times Y) = \partial U_n \cap \partial U_{n-1} = \emptyset\]
  bo $\cl{U_{n-1}} \subset U_n$.
  
  Zachodzi również zgodność pozostałych wzorów. Niech $(x, z) \in \partial V_{n-1} \times X$. Wówczas $(x, \star_n) \in \partial U_{n-1}$ a więc, jak poprzednio, $\lambda(x, z, \star_{n+1}) = \lambda(x, \star_n) = n$ - co daje zgodność wzorów lub $d_{n-1}(x, s_{n-1} A) = \frac{1}{n-1}$ - co wyklucza należenie do $A_n$, ponieważ $(x, z) \in A_n$ wymusza $d_{n}((x, z), s_{n} A) = 0$, a więc i $d_{n-1}(x, s_{n-1} A) = 0$.
  
  Podobnie pokazujemy zgodność $\partial V_n$ z $A_n$.
  
  Ostatecznie do $\sigma_n$ doklejamy wzdłuż $\partial V_{n-1} \times X$ zgodną z nią funkcję $\sigma_{n-1}$ określoną produktowo na zbiorze $\cl{V_{n-1}} \times X$. Z twierdzenia Tietzego znów przedłużamy $\sigma_n$ otrzymując funkcję $\sigma_n$ określoną na $\cl{V_n}$.
  
  Funkcję $\rho_n$ definiujemy jak poprzednio:
  
  \[\rho_n y := \sigma_n s_n y\]
  
  To kończy nasze postępowanie indukcyjne.

  Spójrzmy teraz na jakim zbiorze pozostaje nam określić funkcję: $Y \setminus \bigcup_{i=1}^\infty \cl{U_i}$. Oczywiście $\bigcup_{i=1}^\infty \{y\ |\ \lambda(y) < n+1\} = \{y\ |\ \lambda(y) < \infty\}$. Przyjrzyjmy się uważniej drugiej części, tj.: $\bigcup_{i=1}^\infty \{y\ |\ d_n(s_n y, s_n A) > \frac{1}{n}\}$. Twierdzimy, że jest ona równa $Y \setminus A$.
  
  Istotnie. Niech $d_n(s_n y, s_n A) \leq \frac{1}{n}\ \forall n$.
  
  Istnieje wówczas ciąg $(a_n)_{n=0}^\infty \in A^\mathbb{N}$, taki że:
  
  \[d_n(s_n y, s_n a_n) < \frac{2}{n}\ \forall n \in \mathbb{N}_1\]
  
  Niech $\varepsilon > 0$, niech $n_0$ będzie tak duże, by $\sum_{i=n_0}^\infty \frac{1}{2^i}< \frac{\varepsilon}{3}$ oraz $\frac{2}{n_0} < \frac{\varepsilon}{3}$. Wówczas dla dowolnego $n \geq n_0$:
  
  \[d(y, a_n) \leq d(y, (s_n y, \star_{n+1})) + d((s_n y, \star_{n+1}), (s_n a_n, \star_{n+1})) + d((s_n a_n, \star_{n+1}), a_n) \leq \frac{\varepsilon}{3} + d_n(s_n y, s_n a_n) + \frac{\varepsilon}{3} \leq \varepsilon\]
  
  Co oznacza, że $a_n \rightarrow y$ i tym samym $y \in \cl{A} = A$.
  
  Zatem jedynym miejscem poza $\bigcup_{i=1}^\infty U_i$ jest $\{y \in A\ |\ \lambda(y) = \infty\}$.
  
  Określamy więc pożądaną funkcję sterującą $\rho$ jako:
  
  \[\rho(x) := 
    \begin{cases}
      \rho_n(x),&\mbox{dla } x \in U_n \\
      \infty,&\mbox{ w przeciwnym razie}
    \end{cases}
  \]
  
  Wykażemy teraz, że funkcja $\rho$ ma postulowane własności.
  
  Poprawna określoność wynika z własności \ref{induction-2} a ciągłość bezpośrednio z własności \ref{induction-3}. Widzimy również, że funkcja $\rho$ poza $A$ jest wszędzie skończona. Z własności \ref{induction-1} wynika natomiast zgodność z $\lambda$ na $A$.
  
  Pozostaje sprawdzić, że $\rho$ jest funkcją sterującą. Istotnie, niech $\rho(y) < n+1$ i $s_n(y) = s_n(x)$. Wówczas $y \in U_n$. Ale $\rho(y) = \rho_n(y) = \sigma_n(s_n y) = \sigma_n(s_n x) = \rho_n(x) = \rho(x)$
\end{proof}
\end{thm}


\begin{cor} \label{cor:steering-finite}
  Niech $A$ będzie domkniętym podzbiorem $X$. Wówczas istnieje funkcja sterująca $\lambda$ taka, że $\lambda(y) = \infty$ wtedy i tylko wtedy gdy $y \in A$.
  
  \begin{proof}
    Niech $\mu: Y \ni y \rightarrow \infty \in [1, \infty]$ będzie funkcją sterującą. Z poprzedniego twierdzenia poprawiamy funkcję $\mu$ na $A$, otrzymując $\lambda$ o żądanych własnościach: $\lambda|A = \mu|A = \infty$ oraz $\lambda(y) < \infty$ dla $y \in Y \setminus A$.
  \end{proof}
\end{cor}

\begin{lem}
  Dla każdej liczby naturalnej większej lub równej $1$ zachodzi elementarna nierówność:
  \[4(n+1) \leq 2^{8n}\]
  
  \begin{proof}
    Dla $n = 1$ nierówność przyjmuje postać: $4(1+1) = 8 \leq 256 = 2^8$.
    Załóżmy, że nierówność jest prawdziwa dla $n \geq 1$. Wówczas:
    
    \[4(n+2) \leq 4(n+1) + 4 \leq 2^{8n} + 4 \leq 2^{8n} + 2^{8n} = 2^{8n + 1} \leq 2^{8(n+1)}\]
    
    Co kończy dowód indukcyjny.
  \end{proof}
\end{lem}

\begin{cor}
  Niech $A$ będzie domkniętym podzbiorem $X$. Wówczas istnieje funkcja sterująca $\mu$ taka że $\rho(y) = \infty$ wtedy i tylko wtedy gdy $y \in A$ oraz:
  
  \[2^{\rho(y)} \geq \frac{4}{d(y,A)}\ \forall y \in Y \setminus A\]
  
  \begin{proof}
    Niech $\lambda$ będzie funkcją sterującą z wniosku \eqref{cor:steering-finite}. Określamy: $\rho := 8 \cdot \lambda$.
    Niech $y \in Y \setminus A$. Wówczas istnieje $n \in \mathbb{N}_1$ takie, że $y \in U_{n+1} \setminus U_n$. Ale skoro $\mu = \infty$, to $U_n = \{y \in Y\ |\ d_n(s_n y, s_n A) > \frac{1}{n}\}$, więc ze względu na $y \in U_{n+1}$:
    
    \[d(y, A) \geq d_{n+1}(s_{n+1}(y), s_{n+1} A) > \frac{1}{n+1}\]
    
    Z drugiej strony z warunku \eqref{induction-3} otrzymujemy $\lambda(y) \geq n$, czyli $\rho(y) \geq 8n$.
    
    Zatem:
    \[\frac{4}{d(y,A)} \leq \frac{4}{\frac{1}{n+1}} = 4(n+1) \leq 2^{8n} \leq 2^{8 \rho(n)}\]
    
    Co należało pokazać.
  \end{proof}
\end{cor}


