\section{Homotopia spychająca}
\begin{df}
  Niech $X$ będzie przestrzenią topologiczną, $Y := \prod_{i=1}^\infty X$. Homotopię $f: Y \times X \times [1,\infty] \rightarrow Y$ nazwiemy spychającą jeśli spełnione są następujące warunki:
  \begin{enumerate}
    \item \label{displacement-proj} $s_n f_t(y,x) = s_n(y)$, dla $n \leq t$
    \item \label{displacement-infty} $f_\infty(y,x) = y$, dla $(y,x) \in Y \times X$
    \item \label{displacement-homeo} $F(y,x,t) := (f_t(y,x), t)$ jest homeomorfizmem między $Y \times X \times [1,\infty)$ a $Y \times [1,\infty)$
  \end{enumerate}
\end{df}

\begin{lem}
  Niech $(f_t)_{0 \leq t \leq 1}$ będzie homotopią spychającą. Wówczas odwzorowanie:
  \[Y \times [1,\infty) \ni (y, t) \rightarrow f_t^{-1}(y) \in (Y \times X)\]
  jest poprawnie określone i ciągłe.
  
  \begin{proof}
    Z warunku \ref{displacement-homeo} wynika, że możemy zastosować warunek $(i)$ lematu \ref{lem:inverse}, kładąc: $X := Y \times X$, $T := [1, \infty)$ oraz $Y := Y$. Wówczas warunek $(ii)$ lematu mówi nam, że:
    
    \[Y \times [1,\infty) \ni (y, t) \rightarrow f_t^{-1}(y) \in (Y \times X)\]
    
    Co chcieliśmy pokazać.
  \end{proof}
\end{lem}

\begin{thm}[O istnieniu homotopii spychającej]
  \label{thm:displacement-homotopy}
  Jeśli $X$ ma RIP, to na $Y$ można skonstruować homotopię spychającą.
  \begin{proof}
    Niech $(g_t)_{0 \leq t \leq 1}$ będzie izotopią odbijającą na $X$. Dla $n \in \mathbb{N}_1$ i $t \in [0,1]$ definiujemy:
    \[f_{n+t}(y,x) := (y_1, \ldots, y_n, g_t(x, y_{n+1}), y_{n+2}, \ldots)\]
    Dalej definiujemy, zgodnie z wzorem \ref{displacement-infty}:
    \[f_\infty(y,x) = y\]
    
    Funkcja ta jest w jasny sposób poprawnie określona na zbiorze $Y \times X \times ([1, \infty) \setminus \mathbb{N}_1)$. W punktach $Y \times X \times \mathbb{N}_1$ obowiązują natomiast dwa wzory:
    \[f_{(n+1) + 0}(y,x) = (y_1, \ldots, y_n, y_{n+1}, g_0(x, y_{n+2}), y_{n+3}, \ldots)\]
    oraz
    \[f_{(n+0) + 1}(y,x) = (y_1, \ldots, y_n, g_1(x, y_{n+1}), y_{n+2}, y_{n+3}, \ldots)\]
    które jednak ze względu na własności izotopii odbijającej:
    \[g_0(x, y_{n+2}) = (x, y_{n+2}) \mbox{ i } g_1(x, y_{n+1}) = (y_{n+1}, x)\]
    zadają funkcję w ten sam sposób. Z powyższego wzoru widzimy, że warunek \ref{displacement-proj} jest spełniony.
    
    Pozostaje więc zaobserwować ciągłość w punktach $Y \times X \times \{\infty\}$. Istotnie, weźmy $(u_n, (y_n, x_n)) \rightarrow (\infty, (y,x))$, $n \rightarrow \infty$. Niech $m \in \mathbb{N}_1$ oraz $n_0$ takie, że $u_n \geq m$ dla $n \geq n_0$. Wówczas:
    \[s_m f_{u_n}(y_n, x_n) = s_m(y_n) \rightarrow s_m(y) = s_m f_\infty (y,x)\]
    
    Co daje ciągłość funkcji $f$.
    
    Spełniony jest również warunek \ref{displacement-homeo}. Istotnie, wyznaczmy funkcję odwrotną:
    \[f_{n+t}(y,x) = (y_1, \ldots, y_n, g_t(x, y_{n+1}), y_{n+2}, \ldots) = z\]
    
    Wówczas:
    \begin{align*}
      y_1 &:= z_1 \\
      & \vdots \\
      y_n &:= z_n \\
      (x, y_{n+1}) &:= g_t^{-1}(z_{n+1}, z_{n+2}) \\
      y_{n+2} &:= z_{n+3} \\
      y_{n+3} &:= z_{n+4} \\
      & \vdots
    \end{align*}

    Zatem funkcję odwrotną należy określić jak następuje:
    \[h_{n+t}(z) := ((z_1, \ldots, z_n, p_2 g_t^{-1}(z_{n+1}, z_{n+2}), z_{n+3}, \ldots), p_1 g_t^{-1}(z_{n+1}, z_{n+2}))\]
    
    gdzie $p_1, p_2:  X \times X \rightarrow X$ są projekcjami na odpowiednio pierwszą i drugą współrzędną, $t \in [0,1]$, $n \in \mathbb{N}_1$. Aby upewnić się, że funkcja $h$ jest ciągła należy jeszcze zauważyć, podobnie jak dla $f$, że poniższe wzory dają w istocie ten sam wynik:
    
    \begin{align*}
    h_{(n+0)+1}(z) &= ((z_1, \ldots, z_n, p_2 g_1^{-1}(z_{n+1}, z_{n+2}), z_{n+3}, \ldots), p_1 g_1^{-1}(z_{n+1}, z_{n+2})) \\
    &= ((z_1, \ldots, z_n, z_{n+1}, z_{n+3}, \ldots), z_{n+2})
    \end{align*}
    
    oraz
    
    \begin{align*}
    h_{(n+1)+0}(z) &= ((z_1, \ldots, z_n, z_{n+1}, p_2 g_0^{-1}(z_{n+2}, z_{n+3}), z_{n+4}, \ldots), p_1 g_0^{-1}(z_{n+2}, z_{n+3})) \\
    &= ((z_1, \ldots, z_n, z_{n+1}, z_{n+3}, z_{n+4}, \ldots), z_{n+2})
    \end{align*}

  \end{proof}
\end{thm}


\begin{prop}
  Homotopia spychająca spełnia następujący warunek:
  \[\mbox{Jeśli } (z_n, t_n) \rightarrow (z, \infty), \mbox{ to } p_Y f_{t_n}^{-1}(z_n) \rightarrow z\]
  \begin{proof}
    Niech $z_n \rightarrow z$ i $t_n \rightarrow \infty$. Oznaczmy:
    \[(y_n, x_n) := f_{t_n}^{-1}(z_n)\]
    A więc
    \[f_{t_n}(y_n, x_n) = z_n\]
    Ustalmy $k \in \mathbb{N}_1$ oraz $n_0$ tak duże, że dla $n \geq n_0$ zachodzi $t_n \geq k$. Wówczas z własności \ref{displacement-proj} mamy:
    \[s_k y_n = s_k f_{t_n}(y_n, x_n)\]
    Skąd:
    \[s_k p_Y f_{t_n}^{-1}(z_n) = s_k y_n = s_k z_n \rightarrow s_k z\]
    Z dowolności $k$ otrzymujemy tezę.
  \end{proof}
\end{prop}

