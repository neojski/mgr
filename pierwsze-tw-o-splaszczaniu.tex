\section{Pierwsze twierdzenie o spłaszczaniu}
\begin{thm}
  Niech $X$ będzie przestrzenią topologiczną z RIP, $Y = X^\mathbb{N}_1$, $\cl A = A \subset Y$. Wówczas istnieje homeomorfizm $h: (Y \times X)_A \rightarrow Y$ taki, że $h(a) = a$ dla $a \in A$.
  
  \begin{proof}
    Niech $\lambda$ będzie funkcją sterującą na $Y$, taką że $\lambda(y) = \infty$ wtedy i tylko wtedy gdy $y \in A$.
    Niech $(f_t)_{0 \leq t \leq 1}$ będzie homotopią spychającą $Y \times X \rightarrow Y$
    Definiujemy:
    \[h(y,x) := f_{\lambda(y)}(y,x)\]
    Wówczas dla $a \in A$, $x \in X$ mamy:
    \[h(a,x) = f_{\lambda(a)}(a,x) = f_\infty(a,x) = a\]
    Co więcej, weźmy $y_n \rightarrow a \in A$. Wtedy $\lambda(y_n) \rightarrow \lambda(a) = \infty$, a więc przy ustalonym $k \in \mathbb{N}_1$ i $n_0$ taki, aby $\lambda(y_n) \geq k$ dla $n \geq n_0$ mamy:
    \[s_k h(y_n, x_n) = s_k f_{\lambda(y_n)}(y_n,x_n) = s_k y_n \rightarrow s_k a\]
    Zatem $h(y_n,x_n) \rightarrow a = h(a, x_1)$, więc $h$ reprezentuje funkcję ciągłą $h_0: (Y \times X)_A \rightarrow Y$.
    
    Pokażemy, że $h_0$ jest szukanym homeomorfizmem, a funkcją do niego odwrotną jest $g$ zdefiniowane następująco:
    
    \[g: Y \ni z \rightarrow
    \begin{cases}
      z,& z\in A \\
      f^{-1}_{\lambda(z)}(z),& z \in Y \setminus A
    \end{cases}\]
    
    Istotnie, aby wykazać ciągłość funkcji $g$ wystarczy sprawdzić ciąg $Y \ni z_n \rightarrow a \in A$. Ale skoro $\lambda(a) = \infty$ mamy $\lambda(z_n) \rightarrow \infty$, co z własności funkcji sterujących daje $p_Y f_{\lambda(z_n)}^{-1}(z_n) \rightarrow a$, co z kolei - ze względu na topologię na $(Y \times X)_A$ oznacza, że:
    \[f_{\lambda(z_n)}^{-1}(z_n) \rightarrow a \mbox{ w } (Y \times X)_A\]

    Należy się jeszcze upewnić, że $h$ i $g$ są funkcjami wzajemnie odwrotnymi.
    
    Najpierw wykażemy prostszą równość, mianowicie $gh_0(y,x) = (y,x)$. Jeśli $y \in A$ mamy wprost $gh_0(y,x) = gy = y$. Jeśli natomiast $y \in Y \setminus A$, to:
    \[gh_0(y,x) = f^{-1}_{\lambda(y)} f_{\lambda(y)}(y,x) = (y,x)\]
    
    Aby wykazać równość $h_0g(z) = z\ \forall z \in Y$, zauważmy następującą własność:
    
    \[\lambda(p_Y f_{\lambda(z)}^{-1}(z)) = \lambda(z)\ \forall z \in Y \setminus A\]
    
    Istotnie, niech
    \[(y,x) := f_{\lambda(z)}^{-1}(z)\]
    Wówczas
    \[z = f_{\lambda(z)}(y,x)\]
    Ale $\lambda(z) < \infty$, więc $\exists n \in \mathbb{N}_1: \lambda(z) \in [n, n+1)$. Ale wtedy z własności homotopii spychającej dla $n \leq \lambda(z)$.
    \[s_n z = s_n f_{\lambda(z)}(y,x) = s_n y\]
    Skoro $\lambda(z) < n+1$, to z własności funkcji sterującej $\lambda$ otrzymujemy:
    \[\lambda z = \lambda y\]
    Co razem z równością $y = p_Y f_{\lambda(z)}^{-1}(z)$ daje wspomnianą własność.
    
    Teraz możemy wrócić do dowodu. Zauważmy, że dla $z \in A$ znów nie ma problemu: $h_0g(z) = h_0(z) = z$. Natomiast dla $z \in Y \setminus A$ mamy:
    \[h_0g(z) = h_0(f_{\lambda(z)}^{-1}(z)) = f_{\lambda(p_Y f_{\lambda(z)}^{-1}(z))}(f_{\lambda(z)}^{-1}(z)) = f_{\lambda(z)}(f_{\lambda(z)}^{-1}(z)) = z\]
    
    Tym samym wykazaliśmy, że $h_0$ jest homeomorfizmem pomiędzy $(Y \times X)_A$ a $Y$ identycznościowym nad $A$.
  \end{proof}
\end{thm}


Bezpośrednio z dowodu wynika następujący wniosek:
\begin{cor}
  Niech $X$ będzie przestrzenią metryczną z RIP, $Y = X^{\mathbb{N}_1}$, $\cl{A} = A \subset Y$.
  Wówczas dla dowolnej funkcji sterującej $\lambda$, takiej że $\lambda(y) = \infty$ wtedy i tylko wtedy gdy $y \in A$ zachodzi:
  
  \[h: Y \times X \ni (y,x) \rightarrow f_{\lambda(y)}(y,x) = 
    \begin{cases}
      f_{\lambda(y)}(y,x),&y \in Y \setminus A \\
      y,&y \in A
    \end{cases}\]
  
  reprezentuje homeomorfizm $h_0$ pomiędzy $(Y \times X)_A$ na $Y$. Co więcej, $h_0(a) = a$, dla $a \in A \subset (Y \times X)_A$.
\end{cor}

