\section{Wprowadzenie}
W latach 60-tych XIX wieku bardzo intensywnie rozwijano teorię rozmaitości nieskończenie wymiarowych. Używając dość prostych narzędzi jak atlas regularny można wykazać, że każdą spójną nieskończenie wymiarową rozmaitość modelowaną na przestrzeni liniowo-metrycznej homeomorficznej ze swoim przeliczalnym produktem można zanurzyć w sposób domknięty w swojej przestrzeni modelowej. Henderson w pracy \cite{hen} dowodzi używając teorii mikrowiązek, że jest $M$ jest nieskończenie wymiarową ośrodkową rozmaitością modelowaną na przestrzeni Frécheta $F$, to $M\times F$ można zanurzyć w sposób otwarty w przestrzeń $F$. Warto jednak podkreślić, że istnieje --- z dokładnością do homeomorfizmu --- dokładnie jedna ośrodkowa przestrzeń Frécheta. Trudno tutaj o wskazanie konkretnego nazwiska i pracy rostrzygającej o tym wyniku, dlatego warto spojrzeć na rys historyczny podany w \cite{bp}. Dzięki twierdzeniu R. Andersona i R. Schoriego --- o którym mowa w niniejszej pracy --- głoszącym, że każda rozmaitość $M$ topologiczna modelowana na przestrzeni liniowo-metrycznej $Y$ homeomorficznej ze swoim przeliczalnym produktem jest homeomorficzna z $M\times Y$, otrzymujemy niezwykły wynik: każda ośrodkowa nieskończenie wymiarowa rozmaitość modelowana na ośrodkowej przestrzeni Hilberta $H$ (a więc i na dowolnej ośrodkowej przestrzeni Frécheta) zanurza się jako otwarty podzbiór swojej przestrzeni modelowej.

Niniejsza praca jest opracowaniem drogi dowodowej przedstawionej w książce \cite{bp}. Rozdział \ref{chap:wlasnosci} zawiera fakty i twierdzenia z ogólnej teorii przestrzeni topologicznych, metrycznych oraz teorii rozmaitości nieskończenie wymiarowych. W rozdziale \ref{chap:narzedzia} przedstawiamy główne narzędzia, które wykorzystywane są w głównym twierdzeniu. Warto wspomnieć, że teoria izotopii odbijających pochodzi od R. Wonga \cite{won}. Rozdział \ref{chap:twierdzenie} zawiera dowód zapowiadanego twierdzenia poprzedzony podobnym wynikiem uzyskanym dla samej przestrzeni modelowej sprytnie przeniesionym na rozmaitość z użyciem specjalnego atlasu.