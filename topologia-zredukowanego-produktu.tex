\section{Topologia zredukowanego produktu}
\begin{df}
  Niech $M$, $N$ będą przestrzeniami topologicznymi, niech $A$ będzie podzbiorem domkniętym przestrzeni $M$. Definiujemy przestrzeń $(M \times N)_A := (M \setminus A) \times N \cup A$ iloczynu przestrzeni $M$ i $N$ zredukowanego nad $A$ z następującą topologią.
  
  Niech
  
  $$p: (M \times N)_A \rightarrow M$$
  
  będzie odwzorowaniem określonym następująco:
  \begin{enumerate}
   \item $p(m, n) := m$, dla $m \in M \setminus A$
   \item $p(a) := a$, dla $a \in A$
  \end{enumerate}
  
  Topologię na zbiorze $(M \times N)_A$ określamy poprzez bazę złożoną ze zbiorów dwu rodzajów:
  \begin{enumerate}
   \item $U \times V$, gdzie $U \subset M \setminus A$, $U$ otwarty w $M$, $V$ otwarty w $N$
   \item $m^{-1}(U)$, gdzie $U$ jest otwarty w $M$
  \end{enumerate}
\end{df}


\begin{prop}
  Niech $M$, $N$ i $Y$ będą przestrzeniami topologicznymi, $A$ domknięty podzbiór $M$, $n_0 \in N$. Wówczas funkcję $f_0: (M \times N)_A \rightarrow Y$ możemy zareprezentować przez funkcję $f: M \times N \rightarrow Y$ stałą nad $A$, tzn. $N \ni n \rightarrow f(a, n) \in Y$ jest stała przy każdym ustalonym $a \in A$. Funkcje te można wzajemnie odtwarzać wzorami:
  \begin{enumerate}
   \item $f_0(m,n) := f(m,n)$, gdy $(m,n) \in (M \setminus A) \times N$, $f_0(a) := f(a, n_0)$, gdy $a \in A$
   \item $f(m,n) := f_0(p(m,n))$, gdy $(m,n) \in M \times N$
  \end{enumerate}

  
  Co więcej, $f_0$ reprezentuje funkcję ciągłą $f$ wtedy i tylko wtedy gdy $\forall (m_i, n_i)_{i=1}^\infty \in (M \times N)^{\mathbb{N}_1}:$
  $$m_i \rightarrow a \in A \mbox{ pociąga } f(m_i, n_i) \rightarrow f(a, n_1)$$
  
  \begin{proof}
    Niech $(m_i, n_i)_{i=1}^\infty \in (M \times N)^{\mathbb{N}_1}$ i $m_i \rightarrow a \in A$.
    
    Wówczas z definicji topologii na $(M \times N)_A$ zachodzi $(m_i, n_i) \rightarrow a$, ponieważ otoczenia $a$ są postaci $p^{-1}(U)$ dla $U$ otoczenie $a$. A więc wobec ciągłości $f_0$ otrzymujemy: $f(m_i, n_i) = f_0 p(m_i, n_i) = f_0(m_i, n_i) \rightarrow f_0(a) = f(a, x_1)$.
    
    Z drugiej strony, wykażemy ciągłość $f_0$. Wystarczy sprawdzić przypadek $(m_i, n_i) \rightarrow a \in A$. Ale wówczas z definicji topologii zredukowanego produktu $m_i \rightarrow a \in A$, co z założenia daje $f_0(m_i, n_i) = f(m_i, n_i) \rightarrow f(a, x_1) = f_0(a)$, czyli ciągłość $f_0$.
  \end{proof}
\end{prop}

